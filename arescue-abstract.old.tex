
\label{abstract}

While virtualization-based systems become a reality, an important
issue is that of virtual machine migration-enabled consolidation 
and dynamic resource provisioning. Mutually communicating virtual machines,
as part of migration and consolidation strategies, may get colocated on
the same physical machine or placed on different machines. In this 
work, we argue the need for network affinity-awareness not only in 
placement but also in resource provisioning for virtual machines.
First, we empirically quantify the resource savings
due to colocation of communicating virtual machines. 
We also discuss the increase in resource usage due to dispersion of
previously colocated virtual machines. 
Next, we build models
based on different resource-usage micro-benchmarks to predict
the resource usages when transitioning from non-colocated placements to
colocated placements and vice-versa. 
These resource usage prediction models are usable along-with consolidation
and migration procedures to determine requirements of VMs in 
colocated and non colocated scenarios. Via extensive experimentation, 
we evaluate the applicability of our models for synthetic and benchmark
application workloads. We find that the models have high prediction 
accuracy\textemdash{}90th
percentile prediction error within 3\% absolute CPU usage for both
synthetic and application workloads.

%Public cloud compute services use virtualization technologies for 
%resource sharing, and need optimal capacity planning/management for 
%economical utilization of hardware infrastructure. Instead 
%of leasing fixed amount of resources, the cloud provider can benefit
%from multiplexing resources among multiple virtual machines via 
%dynamic resource provisioning, as per load requirements. Similar 
%considerations apply in the ``private'' cloud setting too, though
%the cloud provider and cloud user are one and the same. Server 
%consolidation enabled by dynamic resource provisioning can
%reduce server sprawl and bring down cooling costs.
%
%In this work, we argue the need for network affinity-aware resource provisioning
%for virtual machines (VMs), with the claim that when inter-VM communication occurs on a
%single physical machine (i.e. intra-PM communication), the subsequent resource
%(network as well as CPU) requirement is lesser in comparison to that for
%inter-PM communication. We demonstrate via controlled experiments that there
%are ``CPU savings'' to be had, when virtual machines are \textit{colocated},
%versus when they are \textit{non colocated}. 


%We further develop a model to 
%provide a mapping from the individual resource utilization profiles of non
%colocated VMs, to their colocated resource usage profile. This model will 
%take as input, various resource utilization metrics corresponding to 
%execution of VMs on two different PMs and output an estimate of CPU resource 
%required for co-hosting both of them on a single PM. In order to make the 
%model network-affinity aware, the network traffic metrics are separated 
%out into \textit{affine} and \textit{non-affine} traffic, where "affine" traffic refers 
%to traffic exchanged between the two VMs and "non-affine" traffic refers 
%to all other network traffic to/from the VMs. This constructed model
%is generic and can be used to estimate the colocated resource usage of 
%any given pair of VMs on a given hardware platform. We present our 
%methodology using the Xen 
%virtual machine platform, and our evaluations show that our models efficiently 
%capture the general mapping from non colocated to colocated resource usage 
%profiles. We present some results of experimental evaluation with synthetic 
%workloads and real applications, and justify applicability in server consolidation
%and VM placement algorithms.

