
We are interested in the problem of estimating the %average
CPU utilization of a virtual machine, based on its location relative
to its communicating set of virtual machines.
% when being
% considered for consolidation or migration to a target PM. 
Additionally, the virtual machine should be
able to continue execution of its tasks to meet
specified service level objectives.
% As we demonstrate (in Section \ref{tnsm-benchmark}),
Since CPU is the primary resource affected due to handling network I/O
operations in colocated and dispersed scenarios, the scope of this work 
is restricted to predicting CPU resource usage.

We consider the following scenario for our problem.
A set of applications, each application having several mutually
communicating components or tiers, are
provisioned as VMs in a cluster of inter-connected PMs.
Thus, each VM has network activity, disk activity and CPU utilization. 
Since disk partitions may be 
network-attached (NFS-mounted\nomenclature{NFS:}{Network File System}\index{NFS} or 
SAN\nomenclature{SAN:}{Storage Area Network}\index{SAN} or
NAS\nomenclature{NAS:}{Network-attached Storage}\index{NAS} appliance), disk activity of
guest VMs\index{VM} may also manifest as network traffic at their host PM\index{PM}.
In this setting,
pro-actively or reactively, a decision process may decide
to move/migrate a subset of VMs to meet dynamic resource
requirements or to consolidate VMs on fewer PMs,
and a vital input to this decision process
is the resource requirements of the VM on the target machine
after migration.

Given a set of virtual machines and their current resource utilization 
levels, our aim is to predict CPU resource required by 
the virtual machine on target host after migration. Additionally,
the virtual machine should be able 
to support the \emph{same load level} as on the source physical machine.
Memory is assumed to be not a bottleneck in our placement configurations,
and the VMs are assumed to maintain their intrinsic resource
utilization levels towards maintaining the SLA guarantees.

In~\cite{virtual-putty}, there is allusion to the possibility 
of change in resource requirements upon a change in the
hosting scenario of two communicating VMs\textemdash{}however, 
empirical quantification is lacking. As part of our work,
we perform a detailed benchmarking exercise to empirically 
demonstrate that there is indeed a difference in CPU utilization of
communicating VMs when their hosting scenario changes between
\textit{colocated} and \textit{dispersed}. The difference in CPU usage for
intra-PM (due to VM colocation) and inter-PM (due to VM dispersion)
network communication forms the motivation for
affinity-aware CPU usage modeling.
% estimating the CPU requirement upon migration. 
Benchmarking for both Xen and KVM virtualization environments is presented 
in the next section and 
% when the current CPU usage and other resource usage
% profiles are known, is that intra-PM (due to colocation) and 
% inter-PM (due to dispersion) network
% communication incur different overheads (as already discussed
% in Section \ref{tnsm-intro}).
% Each VM's CPU utilization is characterized by its
% own CPU usage (CPU utilized by DomU), its affine and non-affine
% network traffic, and its disk read/write activity.
% Dom0 CPU utilization occurs on account of the guest or application
% VM's disk and network
% I/O, and is characterized by the hosted VMs' resource usages.
% Based on whether the VMs are being consolidated or separated
% out, the net CPU usage will decrease or increase.
towards CPU requirement prediction, we build models for both the 
colocation and dispersion scenarios, which we describe in detail 
in the following sections.
