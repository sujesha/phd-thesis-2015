
Co-hosting of virtualized applications results in similar content 
across multiple blocks on disk, which are fetched into memory (the 
host's page cache). Content similarity can be harnessed both to
avoid duplicate disk I/O requests that fetch the same content 
repeatedly, as well as to prevent multiple occurrences of duplicate 
content in cache.
Typically, caches store the most recently or frequently
accessed blocks to reduce the number of disk read accesses. 
These caches are referenced by block number, and can not recognize content 
similarity across multiple
blocks. Existing work in memory deduplication merges cache pages after
multiple identical blocks have already been fetched from disk into cache, 
while existing work in I/O deduplication reserves a portion of the
host-cache to be maintained as a content-aware cache.
We propose a disk I/O reduction system for the virtualization environment
that addresses the dual problems of duplicate I/O and duplicate content 
in the host-cache, without being invasive.

We build a disk read-access optimization called DRIVE, that 
identifies content similarity across multiple blocks, and performs
hint-based read I/O redirection to improve cache effectiveness,
thus reducing the number of disk reads further.
% than other systems.
A metadata store is maintained based on the virtual machine's disk
accesses and implicit caching hints are collected 
for future read I/O redirection.
The read I/O redirection is performed from within the virtual 
block device in the virtualized system, to manipulate the entire 
host-cache as a content-deduplicated cache implicitly.
Our trace-based evaluation using a custom simulator, reveals that 
DRIVE always performs equal to or better than the Vanilla system, 
achieving up to 20\% better cache-hit ratios and reducing the
number of disk reads by up to 80\%. The results also indicate that
our system is able to achieve up to 97\% content deduplication in
the host-cache.
%, using only the implicit hints collected regarding 
%host-cache state.
