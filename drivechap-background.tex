
%\subsection{Existing I/O Reduction Techniques}
% \label{sec:iodedupchap-background}
% 
This section presents an overview
of existing techniques for I/O reduction in virtualized data centers,
then presents other page cache management techniques and their
shortcomings in virtualized environment.

\subsection{Virtualization-based I/O Reduction Techniques}
Reduction of disk read accesses can be achieved by 
better host cache usage (for example, use as a content-cache~\cite{iodedup}),
better cache management 
(insertion and eviction) policies~\cite{outperforming-LRU}, 
and content-based deduplication techniques.
Prior work~\cite{cooperative, VMAR, capo} to achieve I/O reduction in 
virtualized data centers focuses on content-based deduplication techniques.
SeaCache\cite{cooperative} deploys a de-duplication
system on the shared storage server in a virtualized data center, and
designs efficient cache eviction algorithms to aggregate distributed
host caches as a unified cache. Thus, it can be used
complementarily with host cache management techniques.

VMAR\cite{VMAR} builds a deduplication map at VM instantiation
time, and redirects read requests for I/O reduction.
Since a long-running application will write many blocks and subsequently
read them again, the deduplication map should ideally be updated for
higher efficiency. However, VMAR's choice to not gather the deduplication 
map dynamically, implies that many
%write requests by
%VMs are not captured on-the-fly,
% for deduplication, hence 
deduplication opportunities may be lost in a long-running application.
Capo\cite{capo} exploits the fact that all VMs are instantiated
from a relatively small number of ``golden images'' and generates a bitmap,
which is then used to eliminate duplicate read requests. However, it
can not detect duplicate requests for data outside of golden images,
for example, customized VMs and workloads of
virtualized applications.
In our work, we build metadata regarding virtual disk data while
the virtual machine (and its hosted application) is executing, and hence
the resulting I/O deduplication can detect duplicates in customized
VM workloads as well.

\subsection{Host Cache Management Techniques}
Memory deduplication~\cite{satori, memorybuddies, difference-engine, singleton}
refers to storing only one copy of duplicate content in memory to 
save memory space and improve cache effectiveness. 
In I/O deduplication~\cite{iodedup}, the aim is to avoid disk I/O requests
that fetch duplicate content into cache. Hence, though both
memory deduplication and I/O deduplication are basically host cache
management techniques\cite{cooperative}, the difference is that
memory deduplication is performed after multiple blocks with duplicate
content have been fetched into memory, whereas I/O deduplication seeks to
avoid block fetches altogether if the same content is already present
in memory. In our work, we seek to improve host cache management via
I/O redirection \& deduplication strategies.

%\subsubsection{Memory Deduplication}
The work in \cite{iodedup} demonstrates that a content-cache 
is better than a block-cache for read-only workloads,
due to the inherent content similarity.
However, it does not present any study of the effects of the above
split-cache approach on overall cache effectiveness.
Specifically,
write requests attain a higher number of cache hits in a block-cache rather
than a content-cache~\cite{iodedup} whereas read requests attain more 
cache hits in a content-cache. 
Since IODEDUP has both a \textit{block-cache} and a \textit{content-cache}, 
it is essential to explore the holistic effects of reserving a fraction of the
block-cache to be used as a content-cache. This is especially
crucial because content-cache benefits may be read/write 
request-mix dependent, as explained next.

The split-cache approach of \cite{iodedup} introduces two problems:
(i)~cache inclusiveness problem, i.e.,
same block present in both the block-cache and the content-cache, and
(ii)~content-cache sizing problem, i.e., determining what portion of 
the host's cache should be maintained as a content-cache
for maximum performance benefits.

To demonstrate these problems, we present results from a simulation experiment
using the traces from two virtualized web-servers hosting 
webmail proxy and course management system (called \textit{webvm}),
available at~\cite{iodedup-online}. This study is presented in the next 
section.







This section presents an overview
of existing techniques for I/O reduction in virtualized data centers,
then presents other page cache management techniques and their
shortcomings in virtualized environment.

\subsection{Virtualization-based I/O reduction techniques}
Reduction of disk read accesses can be achieved by 
better host cache usage (for example, use as a content-cache~\cite{iodedup}),
better cache management 
(insertion and eviction) policies~\cite{outperforming-LRU}, 
and content-based deduplication techniques~\cite{VMAR}.

Prior work~\cite{cooperative, VMAR, capo} to achieve I/O reduction in 
virtualized data centers focuses on content-based deduplication techniques.
SeaCache\cite{cooperative} deploys a de-duplication
system on the shared storage server in a virtualized data center, and
designs efficient cache eviction algorithms to aggregate distributed
host caches as a unified cache. Thus, it can be used
complementarily with host cache management techniques.
VMAR\cite{VMAR} builds a deduplication map at VM instantiation
time, and redirects read requests for I/O reduction.
Since a long-running application will write many blocks and subsequently
read them again, the deduplication map should ideally be updated for
higher efficiency. However, VMAR's choice to not gather the deduplication 
map dynamically, implies that many
%write requests by
%VMs are not captured on-the-fly,
% for deduplication, hence 
deduplication opportunities may be lost in a long-running application.
Capo\cite{capo} exploits the fact that all VMs are instantiated
from a relatively small number of ``golden images'' and generates a bitmap,
which is then used to eliminate duplicate read requests. However, it
can not detect duplicate requests for data outside of golden images,
for example, customized VMs and workloads of
virtualized applications.

In our work, we build metadata regarding virtual disk data while
the virtual machine (and its hosted application) is executing, and hence
the resulting I/O deduplication can detect duplicates in customized
VM workloads as well.

\subsection{Cache management techniques for I/O reduction}
Memory deduplication~\cite{vmware-esx-server, satori, memorybuddies, 
difference-engine, singleton}
refers to storing only one copy of duplicate content in memory to 
save memory space and improve cache effectiveness. 
In I/O deduplication~\cite{iodedup}, the aim is to avoid disk I/O requests
that fetch duplicate content into cache. Hence, though both
memory deduplication and I/O deduplication are basically cache
management techniques\cite{cooperative}, the difference is that
memory deduplication is performed after multiple blocks with duplicate
content have been fetched into memory, whereas I/O deduplication seeks to
avoid block fetches altogether if the same content is already present
in memory. In our work, we seek to improve host cache management via
I/O redirection \& deduplication strategies.

\underline{Memory deduplication techniques:} 
Previous work in \cite{vmware-esx-server, difference-engine} has
shown the possibility of saving memory using deduplication
techniques\index{Memory deduplication},
so as to improve the caching performance. However, the content
similarity identification in these approaches is done via periodic
scanning of guest VM memory. Memory scanning is beneficial for 
memory deduplication only when done at a high scan-rate, however,
this results in higher CPU overheads.
Moreover, because of their overcommitment of memory to guest VMs, 
the Hypervisor has to resort to paging of VM memory in order to 
ensure that all VM's contents are accommodated in memory, and this can
lead to degraded performance.

The work in \cite{satori} develops a memory-deduplication 
system called Satori, in which multiple blocks in cache 
having same content are mapped to a single page, which
is marked as \textit{read-only}. Due to the \textit{read-only} 
setting of shared pages, an attempt to write to it will
cause a page fault, called \textit{copy-on-write-fault}
and the Hypervisor will handle the fault by allocating
and new page frame with the page contents copied. Also,
the page table mappings need to be updated to ensure
that the faulting VM uses the newly allocated page
henceforth.

In our work, we build content-similarity metadata
by dynamic interception\index{Interception} 
of read and write requests within
the virtual disk driver in the VM. Hence, there is
no scanning of memory and related overheads are avoided.
Moreover, an update to the metadata is a ``local''
update as far as the VM's virtual disk driver is
concerned, hence costly page faults and any related
traps to Hypervisor are also unnecessary in our system. Thus, 
our system can achieve efficient cache usage 
with lower overheads than existing memory deduplication 
systems.

%\subsubsection{Memory Deduplication}
\underline{I/O deduplication techniques:}\index{I/O deduplication}
The work in \cite{iodedup} demonstrates that a 
content-cache \index{Content-cache}
is better than a block-cache\index{Block-cache} for read-only workloads,
due to the inherent content similarity.
However, it does not present any study of the effects of the above
split-cache approach on overall cache effectiveness.
Specifically,
write requests attain a higher number of cache hits in a block-cache rather
than a content-cache~\cite{iodedup} whereas read requests attain more 
cache hits in a content-cache. 
Since IODEDUP\index{IODEDUP} has both 
a \textit{block-cache} and a \textit{content-cache}, 
it is essential to explore the holistic effects of reserving a fraction of the
block-cache to be used as a content-cache. This is
crucial because content-cache benefits may be read/write 
request-mix dependent, as explained next.

The split-cache approach of \cite{iodedup} introduces two problems:
(i)~cache inclusiveness problem, i.e.,
same block present in both the block-cache and the content-cache, and
(ii)~content-cache sizing problem, i.e., determining what portion of 
the host's cache should be maintained as a content-cache
for maximum performance benefits.
To demonstrate these problems, we present results from a simulation experiment
using the traces %from two virtualized web-servers hosting 
%webmail proxy and course management system (called \textit{webvm}),
available at~\cite{iodedup-online}. This study is presented in the next 
section.

\subsection{Revisiting disk I/O virtualization}
The de-coupling of the front-end and back-end virtual block drivers
%is beneficial in the following ways.
%First, it 
enables trapping and redirection\index{I/O redirection} 
of guest disk I/O requests before they 
encounter the host page cache.
%Second, it allows access to the virtual block interface for building our
%storage optimization hence allowing our implementation to stay modular
%and pluggable. Third, a non-virtualized host's buffer/page cache can not
%be directly manipulated in a content-deduplication fashion as already
%acknowledged\cite{iodedup}; and virtualization offers a simple alternative to
%easily intercept and redirect I/Os from guest VMs. Finally, since this
%layer is much above the cache and block layer of host, all I/Os
%generated by our deduplication module can still leverage physical
%disk I/O scheduling services at the host as before.
In our work, we exploit this de-coupling of drivers and introduce 
an I/O redirection mechanism such that the host cache in a virtualized
system is implicitly manipulated in a
content-deduplicated fashion without needing any invasive changes
in functioning.
In the next section, we present the design of our I/O reduction 
\& deduplication system called DRIVE (\underline{D}isk I/O 
\underline{R}eduction \underline{i}n 
\underline{V}irtualized \underline{E}nvironments)\index{DRIVE}.
