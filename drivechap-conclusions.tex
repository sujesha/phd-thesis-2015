
I/O reduction refers to reducing the number of disk read accesses by
employing better cache management strategies. Typically, caches are 
referenced by block number and mitigate the necessity to fetch the block
from disk. However, traditional caches can not recognize content similarity
across multiple blocks and hence, the system ends up fetching and storing 
multiple copies of the same content in cache.
Elimination of duplicate read I/O requests which fetch the same 
content repeatedly is referred as I/O Deduplication.
Existing work uses a split-cache approach, with a part of the block-based
cache reserved as a content-based cache. In this work, we demonstrated
that the split-cache approach is sub-optimal, and presented an
I/O reduction system called DRIVE which performs I/O 
redirection to \textit{implicitly} manipulate the whole underlying cache as 
a content-deduplicated cache. Only the VM's own disk access history is
introspected to obtain implicit hints regarding host cache state, to be used
for read I/O redirection.

We performed comparative evaluation by implementing 
prototypes and performed trace-based evaluation in a custom simulator. 
The evaluations showed that, 
%while IODEDUP system reserves a part
%of the total memory to be used as a content-cache, 
the DRIVE system
manipulates the entire available host cache space effectively like
a content-cache, achieving a high content deduplication factor of up to 97\%.
This is the key reason for better performance, with
up to 20\% higher cache-hit ratios,
and up to 80\% higher number of disk reads reduced than the Vanilla system.
