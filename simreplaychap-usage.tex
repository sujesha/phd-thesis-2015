
Listing \ref{lst:usage} shows the basic usage options for the custom
simulator, \textit{simreplay}, in alphabetic order. Each option is
explained below.

\lstset{language=C,
	caption={Listing of simreplay usage},
	label=lst:usage
}
\begin{snippet}
static char usage_str[] =                                               \
        "\n"                                                            \   
        "\t[ -C        : --iodedup-logformat        ] Default: 0\n"     \   
        "\t[ -d <dir>  : --input-directory=<dir>    ] Default: .\n"     \   
        "\t[ -e        : --read-enable              ] Default: Off\n"   \   
        "\t[ -E        : --write-enable             ] Default: Off\n"   \   
        "\t[ -f <file> : --input-file=<file>        ] Default: None\n"  \   
        "\t[ -h        : --help                     ] Default: Off\n"   \   
        "\t[ -m <file> : --mapv2p=<file>            ] Default: None\n"  \   
        "\t[ -O <in-MB>: --overall-RAM-size=<in-MB> ] Default: 1024\n"  \   
        "\t[ -o <in-MB>: --contentcache-size=<in-MB>] Default: 100\n"   \   
        "\t[ -Q        : --Vanilla-replay           ] Default: On\n"    \   
        "\t[ -R        : --DRIVE-replay             ] Default: Off\n"   \   
        "\t[ -T        : --IODEDUP-replay           ] Default: Off\n"   \   
        "\t[ -V        : --version                  ] Default: Off\n"   \   
        "\n";	
\end{snippet}

The option \texttt{iodedup-logformat} indicates that the content information
present in each trace request is in terms of the MD5 hash of the content,
as is the case with the traces \textit{webvm}, \textit{homes} and
\textit{mail} available online at \cite{iodedup-online}. 
This flag is especially useful for parsing of the input trace file.
When the flag is specified, the content field is expected to be 32 
characters (i.e hexadecimal representation of 16-byte MD5 hash). On the 
other hand, if this flag is 
not specified, the original content is expected to be present in the trace. 
For example, if \texttt{iodedup-logformat} is not indicated, the size of the 
content is equal to the value in the \texttt{nbytes} field in the trace
request data-structure.

The options \texttt{input-directory} and \texttt{input-file} are used
to specify the input trace file to be used for simulated I/O replay. The
file contains multiple read/write requests, each in the format described
earlier. Parsing of the file is performed by the "Trace input \& parsing
module" as discussed in Section \ref{sec:simreplay-parsing}.

The options \texttt{read-enable} and \texttt{write-enable} indicate whether
read requests or write requests should be replayed, respectively. If 
\texttt{read-enable} is not specified, read requests replay will not be 
simulated and if \texttt{write-enable} is not specified, write requests 
will not be simulated. Note that, at least one of these two options 
has to be specified otherwise there is no simulation to be done.
By default, both options are disabled, however a typical invocation of
simulator would enable both options so that all read and
write requests present in the trace file would be replayed/simulated.

The option \texttt{mapv2p} can be used to specify the input V2P mapping
indicating the range of physical blocks that map to the address space of 
each VM, as described earlier.
If the trace file has only a single VM's requests, the input V2P
mapping is optional, but is mandatory otherwise.

The option \texttt{overall-RAM-size} indicates the space in 
number of megabytes (MB) to be simulated as the block-cache in the
Vanilla system. By default, the value is 1024 MB, i.e. 1GB block-cache.
This option is also applicable to the IODEDUP system, but in conjunction
with the value specified using the option \texttt{contentcache-size},
as follows. The option \texttt{contentcache-size} specifies what portion
of the \texttt{overall-RAM-size} should be simulated as a content-cache,
and defaults to 100 MB.
This option is valid only for IODEDUP replay, and re-adjusts the configured
size of the block-cache to \texttt{overall-RAM-size} minus 
\texttt{contentcache-size}. It follows that, the value of 
\texttt{contentcache-size} must be less than or equal to 
\texttt{overall-RAM-size}, for successful invocation of the simulator.

\lstset{language=bash,
	caption={Sample output of \texttt{SimReplay} for Vanilla invocation.},
	label=lst:vanilla-sample
}
%\begin{minipage}[t][5cm][b]{0,5\textwidth}
%\begin{Verbatim}[frame=single]
\begin{snippet}
$ cat stats-O1024-webvm-iodedup-sreplay_rw.txt
#READ=3116456, #WRITE=11177702
RAM-size = 1024 (MB)
buffer cache: hits=12303451, misses=1990707, readhits=1471112, writehits=10832339
disk hits=12823046
disk hits read=1645344, writes=11177702
\end{snippet}
%\end{Verbatim}
%\end{minipage}

\lstset{language=bash,
	caption={Sample output of \texttt{SimReplay} for IODEDUP invocation.},
	label=lst:iodedup-sample
}
\begin{snippet}
$ cat stats-O1024-o100-webvm-iodedup-ioreplay_rw.txt 
io-redirections: self-is-leader=813572, self-is-not-leader=1409885
read-responses: compulsory-misses=316045, cache-hits=704111, capacity-misses=2096300
metadata-hit-conversions: deduphits=127156, selfhits=1, dedupmisses=1282729, selfmisses=813571
#READ=3116456, #WRITE=11177702
RAM-size = 1024 (MB)
CCACHE-size = 100 (MB)
buffer cache: hits=11185632, misses=3108526, readhits=576954, writehits=10608678
content metadata: hits=2223457, misses=316045 dirties=0
content cache: hits=127157, misses=8035550
content cache: dedup hits=127156, nondedup hits=1
disk hits=13590047
disk hits read=2412345, writes=11177702
\end{snippet}

\lstset{language=bash,
    caption={Sample output of \texttt{SimReplay} for DRIVE invocation.},
    label=lst:drive-sample
}
\begin{snippet}
cat stats-O1024-webvm-iodedup-freplay_rw.txt 
io-redirections: self-is-leader=1754745, self-is-not-leader=1045666
read-responses: compulsory-misses=316045, cache-hits=2722411, capacity-misses=78000
metadata-hit-conversions: deduphits=1033328, selfhits=1689083, dedupmisses=12338, selfmisses=65662
#READ=3116456, #WRITE=11177702
RAM-size = 1024 (MB)
confided metadata: hits=2800411, misses=316045, mapmisscachehits=0, dirties=0, mapdirtycachehits=0
fcollisions=0, fcollisionstp=3141002, fzeros=0
buffer cache: hits=13635026, misses=659132, readhits=2722411, writehits=10912615
disk hits=11571747
disk hits read=394045, writes=11177702
\end{snippet}

The options \texttt{Vanilla-replay} and \texttt{IODEDUP-replay} indicate
that the simulator is to be invoked to simulate Vanilla execution and
IODEDUP execution, respectively.
Listings \ref{lst:vanilla-sample} and \ref{lst:iodedup-sample} 
show sample outputs of the simulator when used for Vanilla and
IODEDUP invocations, respectively.
Similarly, the option \texttt{DRIVE-replay} indicates that the
simulator is to be invoked to simulate DRIVE execution.
A sample output of DRIVE invocation of the simulator 
is shown in Listing~\ref{lst:drive-sample}.

The sample output of Vanilla invocation in Listing~\ref{lst:vanilla-sample}
has information regarding the number of read and write requests in the trace,
the total cache (RAM) size used, the number of hits and misses in the 
buffer (block) cache, and a classification of the hits into reads hits or
write hits as well. Finally, it also reports the number of requests that
went to disk\textemdash{}in case of reads, this is the number of disk 
reads that happened, and in case of writes, this is the total number of
writes since we assume that all writes are flushed to disk eventually.

The sample output of IODEDUP and DRIVE invocation also have similar fields
as above, however they have some additional fields too, to track the
efficiency achieved by each system. For example, Listing \ref{lst:iodedup-sample}
presents information regarding the configured size of content-cache,
the number of metadata hits incurred, and how many of those were for
reads and writes, each. It also presents information regarding how many
of the metadata hits eventually resulting in content-cache hits and misses.
Additionally, for all of the content-cache hits, it lists how many were hits for
duplicate content and how many were just regular hits.

The \texttt{help} option shows the usage listing 
for the \texttt{SimReplay} tool, and the
\texttt{version} option can be used to identify different versions of
the custom simulator, for future extension.
