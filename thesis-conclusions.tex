
%This will be the final chapter of the thesis.  A brief report of the work carried out shall 
%form the first part of the Chapter.  Conclusions derived from the logical analysis presented in 
%the Results and Discussions Chapter shall be presented and clearly enumerated, each point 
%stated separately.  Scope for future work should be stated lucidly in the last part of the chapter. 


To address the problem of server consolidation or migration to
a target machine (example, for alleviating resource usage 
hotspots~\cite{sandpiper}),
the foremost requirement is to estimate the expected resource
requirements of the virtual machine on the target machine. This
knowledge in turn, can enable us to make the decision of whether
such a migration or server consolidation is advisable.
As part of this work,
we concern ourselves with the Xen virtualization environment
and are interested in predicting the CPU resource requirements
of VMs. In Xen virtualization environment, since the privileged
domain Dom0 needs CPU scheduling to perform IO operations
(network operations and disk operations for NFS-mounted
VM images) on behalf of the VMs, we are also interested in predicting the
corresponding Dom0 CPU resource requirement.
Though we work with Xen, our approach is generic enough
to be applicable to other virtualization solutions.

In this thesis, first we considered the problem of 
server consolidation with an affinity-aware approach. Two VMs are 
said to have \textit{network-affinity} if they 
communicate with each other, and we show that colocation
of communicating VMs on a single PM helps reduce the total CPU usage.
With detailed measurements, we justified the need for
affinity-aware placement and presented our approach to estimate
the colocated CPU usage for dispersed VMs and vice versa. We 
further presented evaluation
of the constructed estimation models with synthetic workloads
and a real application. We observe that the pair-wise models
to predict different CPU usage 
perform well, achieving maximum prediction
error within 2\% absolute CPU utilization. We also applied
the pair-wise models to multi-VM scenarios using a multi-phase
prediction methodology, and evaluation showed maximum
error to be within 2\%.
\\
\\
Next, we considered the problem of improving disk access performance
by using the page cache more efficiently. 
I/O reduction refers to reducing the number of disk read accesses by
employing better cache management strategies. Typically, caches are
referenced by block number and mitigate the necessity to fetch the block
from disk. However, traditional caches can not recognize content similarity
across multiple blocks and hence, the system ends up fetching and storing
multiple copies of the same content in cache.
Elimination of duplicate read I/O requests which fetch the same
content repeatedly is referred as I/O Deduplication.
Existing work uses a split-cache approach, with a part of the location-addressed
cache reserved as a content-addressed cache. In this work, we demonstrated
that the split-cache approach is inefficient, and presented an
I/O reduction system called DRIVE which performs I/O
redirection to \textit{implicitly} manipulate the whole underlying cache as
a content-deduplicated cache. Only the VM's own disk access history is
introspected to obtain implicit hints regarding host cache state, to be used
for read I/O redirection.

We performed comparative evaluation by implementing
prototypes and performed trace-based evaluation in a custom simulator.
The evaluations showed that,
%while IODEDUP system reserves a part
%of the total memory to be used as a content-cache, 
the DRIVE system
manipulates the entire available host cache space effectively like
a content-cache, achieving a high content deduplication factor of up to 97\%.
This is the key reason for better performance, with
up to 20\% higher cache-hit ratios,
and up to 80\% higher number of disk reads reduced than the Vanilla system.
\\
\\
% The efficiency of disk
% cache is improved by manipulating its usage to mimic a content-aware
% cache, by identifying content similarity at the granularity of 
% variable-sized chunks and fetching the same physical block each
% time that same content is requested. This I/O redirection is achieved by
% building a mapping of fixed-size blocks to deduplicated variable-sized
% chunks, where the chunk boundaries are derived based on content and different
% blocks having same content map to the same deduplicated chunk. Once the
% duplicate chunks are identified, redirection of read I/O requests is done
% such that presence of multiple physical blocks in cache containing same 
% content can be avoided to the extent possible. Although
% this can not be guaranteed, intelligent I/O redirection alternatives are to be
% experimented with to judge which is better in general.

%In PROVIDED system, Rabin chunking of virtual-block data is performed and mapping is created from Rabin chunk to virtual block, referred to as the chunk-to-physical mapping. We also need to build the virtual-to-chunk mapping at this time. Once the mappings are created, the system is ready to process block read/write requests. For read requests, fetch blocks (in granularity of chunks) from disk only if not present in cache. Use data from disk write requests to update cache content and keep the mappings coherent. However, disk write requests can not be de-duplicated or avoided.
%Last, we presented brief ideas about the directions for future 
%work as part of this thesis. We also presented a few open directions for
%research which, though interesting and important, are not within the scope
%of this thesis.
