As mentioned earlier, the first component of
this thesis addresses the need for affinity-awareness for predicting CPU
usage in colocated and dispersed placements of mutually communicating
VMs. 
In the previous chapter (Chapter~\ref{chap:thesis-v2v-1}), we presented benchmarking results for
the Xen virtualization environment as well as developed a linear
regression model for predicting the CPU usage upon a change in placement
of a pair of virtual machines (i.e., colocated or dispersed).
In the network benchmarking presented therein, different 
network rates were generated by transmitting packets / segments of
a single size with varying time intervals.
However, we subsequently hypothesized that the
application ``segment size'' of network transmission also has an impact 
on the resulting virtualized CPU usage, and hence in this chapter,
we undertake the task of performing further benchmarking to
prove our claim. Moreover, we develop better (and simpler) models for predicting
the CPU usage, as explained in the following sections.


\section{Motivation} 
\label{sec:2ndchap-intro} 
\input{2ndchap-intro} 

% %Keep related work here instead of in 1st chapter
% \section{Related work}
% \label{sec:2ndchap-related-work} 
% \input{2ndchap-related-work}

%Move background and profiling study into 1st chapter
%\section{Background}
%\label{sec:2ndchap-background} \input{2ndchap-background} 

%\section{Problem Statement}
%\label{sec:2ndchap-problem} \input{2ndchap-problem} 

\section{Network benchmarking with different segment sizes}
\label{sec:2ndchap-benchmark} 
\input{2ndchap-benchmark}

\section{Linear regression modeling for differential CPU usage estimation}
\label{sec:2ndchap-our-approach} 
\input{2ndchap-our-approach} 

\section{Experimental evaluation}
\label{sec:2ndchap-experimental-eval} 
\input{2ndchap-experimental-eval}


\section{Open directions}
\label{sec:2ndchap-open-directions} 
\input{2ndchap-open-directions}
%\input{2ndchap-conclusions}

