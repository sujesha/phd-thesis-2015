In this sub-section, we present a brief description of basic features like
\texttt{relayfs} channels, \texttt{debugfs} file system, \texttt{kprobes}
and \textit{kernel signaling}, which were used for developing the 
block tracing toolkit.

\subsection{Relayfs channels}
The Relayfs\cite{relayfs} filesystem is designed to enable large amounts 
or sustained
data delivery from kernel modules to user space. Basically, it is a 
``channel'' that routes data from the kernel module into a set of
per-CPU buffers that are operated essentially like a file. 
The Relayfs API consists of two sets of API\textemdash{}the first set for 
in-kernel clients, and the second set for userspace tools to 
dump out the data into persistent storage. We used both sets of APIs
to build our toolkit\textemdash{}the first set to build our content-dumping 
kernel module(s), and the second set to build a generic userspace tool
that can be used with any kernel module that uses the in-kernel 
relayfs APIs. Note that, the relay channels work in tandem with the
\texttt{debugfs} filesystem, as explained next.

\subsection{Debugfs filesystem}
Although relay channels route the data from in-kernel to outwards, the 
userspace tools can access the data only if the buffers are exposed
to userspace like a filesystem. This is where the \texttt{debugfs}\cite{debugfs} 
filesystem comes into play. When the in-kernel Relayfs API called 
\texttt{relay\_open()} is invoked, it results in abstract files being
created within the Debugfs filesystem, per CPU. These files can be 
accessed via \texttt{poll()}, \texttt{open()} and 
\texttt{mmap()} from userspace tools.

\subsection{Kprobes}
Kprobes\cite{kprobes} form the basic essence of the block request tracing 
kernel modules. Kprobes provide a minimally-invasive way to dynamically
trap into any kernel routine, and collect debugging information
non-disruptively. Thus, we can trap any kernel address, and specify
a handler to be executed before the original execution begins. There
are three types of kernel probes: 
(i)~\texttt{kprobes}\textemdash{}can be inserted on \textit{any} instruction in the
kernel,
(ii)~\texttt{jprobes}\textemdash{}can be inserted at kernel function entry, and
provides convenient access to function's arguments,
(iii)~\texttt{kretprobes}\textemdash{}inserted to be fired at function exit.
In our kernel modules, we use \texttt{jprobes} at the entry of the
kernel function \texttt{generic\_make\_request} to gain access to its
function arguments, of type \texttt{struct bio}. More details regarding
usage of \texttt{jprobes} in tracing tool implementation, are presented
in the next section.

\subsection{Signaling between kernel and user space}
An example instantiation of how to ``Send signals from kernel to user space'' 
is presented at \cite{kernel-signaling}. We borrow the source code and
tweak it to fit it into our kernel-module and the userspace log-siphoning
program. The basic idea is that the kernel can signal a userspace process
only if its PID is known. Hence, the userspace process is required to 
let its own PID be known/accessible to the kernel, by writing it into a
pre-determined location, such as a file in the \texttt{debugfs} filesystem
described above. Note that, from within a kernel module, it is 
unsafe/disallowed to open() or close() regular files. However, creating,
opening and closing files in the \texttt{debugfs} filesystem is allowed,
as mentioned above. Additionally, the kernel can register handlers to 
specify the functionality to be performed upon read() or write() of
the file created in \texttt{debugfs} filesystem. Thus, by using the
\texttt{debugsfs}, any userspace process's PID can be conveyed to the
kernel-module, so that signals may be sent to that PID from kernel space.
